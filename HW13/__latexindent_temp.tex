%%%%%%%%%%%%%%%%%%%%%%%%%%%%%%%%%%%%%%%%%%%%%%%%%%%%%%%%%%%%%%%%%%%%%%%%%%%%%%%%%%%%
%Do not alter this block of commands.  If you're proficient at LaTeX, you may include additional packages, create macros, etc. immediately below this block of commands, but make sure to NOT alter the header, margin, and comment settings here. 
\documentclass[12pt]{article}
\usepackage[margin=1in, bottom=4.5cm]{geometry}
\usepackage{amsmath,amsthm,amssymb,amsfonts, enumitem, fancyhdr, color, comment, graphicx, environ, scrextend, mathtools, yfonts}
\usepackage[table,dvipsnames]{xcolor}
\usepackage{tikz}  
\usepackage{tikz-3dplot} 
\usepackage{amssymb}
\usepackage{xifthen}
\pagestyle{fancy}
\setlength{\headheight}{65pt}
\newenvironment{problem}[2][Problem]{\begin{trivlist}
\item[\hskip \labelsep {\bfseries #1}\hskip \labelsep {\bfseries
#2.}]}{\end{trivlist}}
\newenvironment{lemma}[2][Lemma]{\begin{trivlist}
\item[\hskip \labelsep {\bfseries #1}\hskip \labelsep {\bfseries #2.}]}{\end{trivlist}}
\newenvironment{sol}
    {\emph{Proof.}
    }
    {
    \qed
    }
\specialcomment{com}{ \color{blue} \textbf{Comment:} }{\color{black}} %for instructor comments while grading
\NewEnviron{probscore}{\marginpar{ \color{blue} \tiny Problem Score: \BODY \color{black} }}
%%%%%%%%%%%%%%%%%%%%%%%%%%%%%%%%%%%%%%%%%%%%%%%%%%%%%%%%%%%%%%%%%%%%%%%%%%%%%%%%%

\newcommand\restr[2]{{% we make the whole thing an ordinary symbol
  \left.\kern-\nulldelimiterspace % automatically resize the bar with \right
  #1 % the function
  \vphantom{\big|} % pretend it's a little taller at normal size
  \right|_{#2} % this is the delimiter
  }}





%%%%%%%%%%%%%%%%%%%%%%%%%%%%%%%%%%%%%%%%%%%%%
%Fill in the appropriate information below
\lhead{Trey Manuszak}  %replace with your name
\rhead{MAT 473: Intermediate Real Analysis II \\ Homework 12: 45, 46, 47, 48} %replace XYZ with the homework course number, semester (e.g. ``Spring 2019"), and assignment number.
%%%%%%%%%%%%%%%%%%%%%%%%%%%%%%%%%%%%%%%%%%%%%

\usepackage{blindtext}
\title{MAT 473: Intermediate Real Analysis II}
\date{April 24, 2020}
\author{Trey Manuszak\\ Arizona State University}


%%%%%%%%%%%%%%%%%%%%%%%%%%%%%%%%%%%%%%
%Do not alter this block.
\begin{document}
%%%%%%%%%%%%%%%%%%%%%%%%%%%%%%%%%%%%%%


\maketitle
\newpage


%Copy the following block of text for each problem in the assignment.
\begin{problem}{49}
Prove Theorem 22.5 parts (iii) and (iv): Let $f,g : \mathbb{R} \to \overline{\mathbb{R}}$ be integrable functions.
\begin{itemize}
  \item[(iii)] Suppose that $f \leq g$. Prove that $\int f \leq \int g$.
  
  \begin{sol}
    Let $f,g : \mathbb{R} \to \overline{\mathbb{R}}$ be integrable functions and $f \leq g$. Then, by linearity, $$0 \leq \int (g - f) = \int g - \int f,$$ which implies $\int f \leq \int g$.
  \end{sol}

  \item[(iv)] Prove that $\left| \int f \right| \leq \int | f |$. 
   
  \begin{sol}
    Let $f : \mathbb{R} \to \overline{\mathbb{R}}$ be integrable Since $\left| f \right|$ is measurable and bounded and $-\left|f \right| \leq f \leq \left| f \right|$, by linearity and monotonicty, we get $$- \int \left| f \right| \leq \int f \int \left| f \right|.$$
  \end{sol} 
\end{itemize}
\end{problem}

\begin{problem}{50}
Compute the value of the limit $$\lim_{n \to \infty} \int_{0}^{\infty}\left( 1 + \frac{x}{n} \right)^{-n}\cos \frac{x}{n}dx.$$ Justify every step of your argument. (Hint: use the monotone convergence theorem, and the theorem on equality of the Reimann and Lebesgue integrals when both apply, to show that $e^{-x}$ is integrable on $[0,\infty]$. Then use the dominated convergence theorem.)
\end{problem}
%%%%%%%%%%%%%%%%%%%%%%%%%%%%%%%%%%%%%%%%
%Do not alter anything below this line.
\end{document}
%%%%%%%%%%%%%%%%%%%%%%%%%%%%%%%%%%%%%%%%%%%%%%%%%%%%%%%%%%%%%%%%%%%%%%%%%%%%%%%%%%%%
%Do not alter this block of commands.  If you're proficient at LaTeX, you may include additional packages, create macros, etc. immediately below this block of commands, but make sure to NOT alter the header, margin, and comment settings here. 
\documentclass[12pt]{article}
 \usepackage[margin=1in, bottom=4.5cm]{geometry}
\usepackage{amsmath,amsthm,amssymb,amsfonts, enumitem, fancyhdr, color, comment, graphicx, environ, scrextend, mathtools}
\usepackage[table,dvipsnames]{xcolor}
\usepackage{tikz}  
\usepackage{tikz-3dplot} 
\usepackage{amssymb}
\usepackage{xifthen}
\pagestyle{fancy}
\setlength{\headheight}{65pt}
\newenvironment{problem}[2][Problem]{\begin{trivlist}
\item[\hskip \labelsep {\bfseries #1}\hskip \labelsep {\bfseries
#2.}]}{\end{trivlist}}
\newenvironment{lemma}[2][Lemma]{\begin{trivlist}
\item[\hskip \labelsep {\bfseries #1}\hskip \labelsep {\bfseries #2.}]}{\end{trivlist}}
\newenvironment{sol}
    {\emph{Proof.}
    }
    {
    \qed
    }
\specialcomment{com}{ \color{blue} \textbf{Comment:} }{\color{black}} %for instructor comments while grading
\NewEnviron{probscore}{\marginpar{ \color{blue} \tiny Problem Score: \BODY \color{black} }}
%%%%%%%%%%%%%%%%%%%%%%%%%%%%%%%%%%%%%%%%%%%%%%%%%%%%%%%%%%%%%%%%%%%%%%%%%%%%%%%%%

\newcommand\restr[2]{{% we make the whole thing an ordinary symbol
  \left.\kern-\nulldelimiterspace % automatically resize the bar with \right
  #1 % the function
  \vphantom{\big|} % pretend it's a little taller at normal size
  \right|_{#2} % this is the delimiter
  }}





%%%%%%%%%%%%%%%%%%%%%%%%%%%%%%%%%%%%%%%%%%%%%
%Fill in the appropriate information below
\lhead{Trey Manuszak}  %replace with your name
\rhead{MAT 473: Intermediate Real Analysis II \\ Homework 10: 37, 38, 39, 40} %replace XYZ with the homework course number, semester (e.g. ``Spring 2019"), and assignment number.
%%%%%%%%%%%%%%%%%%%%%%%%%%%%%%%%%%%%%%%%%%%%%

\usepackage{blindtext}
\title{MAT 473: Intermediate Real Analysis II}
\date{April 10, 2020}
\author{Trey Manuszak\\ Arizona State University}


%%%%%%%%%%%%%%%%%%%%%%%%%%%%%%%%%%%%%%
%Do not alter this block.
\begin{document}
%%%%%%%%%%%%%%%%%%%%%%%%%%%%%%%%%%%%%%


\maketitle
\newpage


%Copy the following block of text for each problem in the assignment.
\begin{problem}{37}
  Let $A_1,A_2,\dots$ be measurable sets, and suppose that $A_1 \subseteq A_2 \subseteq A_3 \subseteq \dots$. Prove that $m(\cup _{n = 1}^{\infty}A_n) = \lim_{n \to \infty}m(A_n)$. (This is called \textit{continuity from below} of Lebesgue measure.) (Hints: use Proposition 16.4 of the notes. It is useful also to remember that $\sum_{n = 1}^{\infty}a_n = \lim_{n \to \infty}\sum_{i = 1}^{n}a_i$.)
\end{problem}
\begin{sol}
  
\end{sol}

\begin{problem}{38}
  Let $A_1,A_2,\dots$ be measurable sets, and suppose that $A_1 \supseteq A_2 \supseteq A_3 \supseteq \dots$. Suppose further that $m(A_1) < \infty$. Prove that $m(\cap_{n = 1}^{\infty}) = \lim_{n \to \infty}m(A_n)$. Be sure to indicate where the finiteness hypothesis is used. (This is called \textit{continuity from above} of Lebesgue measure.) (Hints: as in the previous problem. Also, you will need to consider $B_{\infty} \coloneqq \cap_{n = 1}^{\infty}A_n$.) Give an example of a decreasing sequence of measurable sets of infinite measure for which the above conclusion is false. 
\end{problem}
\begin{sol}
  
\end{sol}

\begin{problem}{39}
  Let $E$ be a measurable set, and let $\epsilon > 0$. Prove that there are an open $U \supseteq E$ and a closed set $F \subseteq E$ such that $m(E \setminus F) < \epsilon$. Here is an outline.

  \begin{itemize}
    \item[(a)] Suppose that $E \supseteq [a,b]$. Use the definition of outer measure to find an open set $U \supseteq E$ with $m(U \setminus E) < \epsilon$.
    \item[(b)] Suppose that $E \subseteq [a,b]$. Apply the previous part to $[a,b] \setminus E$ to prove that there is a closed set $F \subseteq E$ with $m(E \setminus F) < \epsilon$.
    \item[(c)] For the general case let $E_n = E \cap [n,n+1]$ for $n \in \mathbb{Z}$, and apply the previous two parts with $\epsilon4^{-(|n|+1)}$. Use the fact that if $S_n \subseteq T_n$ then $(\cup_nT_n) \setminus (\cup_nS_n) \subseteq \cup_n(T_n \setminus S_n)$.
  \end{itemize}
\end{problem}
\begin{sol}
  
\end{sol}

\begin{problem}{40}
  The \textit{Cantor set}, $C$, is a subset of $[0,1]$ defined as follows. Let $F_0 = [0,1], F_1 = [0,\frac{1}{3}]\cup[\frac{2}{3},1]$, and in general, $F_{n+1}$ is obtained from $F_n$ by deleting the middle open third of each subinterval of $F_n$. (Thus $F_2 = [0,\frac{1}{9}]\cup[\frac{2}{9},\frac{1}{3}]\cup[\frac{2}{3},\frac{7}{9}]\cup[\frac{8}{9},1]$.) Then $C \coloneqq \cap_{n = 1}^{\infty}F_n$. Prove the following:
  \begin{itemize}
    \item[(a)] $F_n$ is the union of $2^n$ pairwise disjoint closed intervals each of length $3^{-n}$.
    \item[(b)] $m(C) = 0$.
    \item[(c)] $C$ is a closed set, $C$ has no isolated points, and the interior of $C$ is empty.  
  \end{itemize}
\end{problem}
\begin{sol}
  
\end{sol}
%%%%%%%%%%%%%%%%%%%%%%%%%%%%%%%%%%%%%%%%
%Do not alter anything below this line.
\end{document}
%%%%%%%%%%%%%%%%%%%%%%%%%%%%%%%%%%%%%%%%%%%%%%%%%%%%%%%%%%%%%%%%%%%%%%%%%%%%%%%%%%%%
%Do not alter this block of commands.  If you're proficient at LaTeX, you may include additional packages, create macros, etc. immediately below this block of commands, but make sure to NOT alter the header, margin, and comment settings here. 
\documentclass[12pt]{article}
\usepackage[margin=1in, bottom=4.5cm]{geometry}
\usepackage{amsmath,amsthm,amssymb,amsfonts, enumitem, fancyhdr, color, comment, graphicx, environ, scrextend, mathtools, yfonts}
\usepackage[table,dvipsnames]{xcolor}
\usepackage{tikz}  
\usepackage{tikz-3dplot} 
\usepackage{amssymb}
\usepackage{xifthen}
\pagestyle{fancy}
\setlength{\headheight}{65pt}
\newenvironment{problem}[2][Problem]{\begin{trivlist}
\item[\hskip \labelsep {\bfseries #1}\hskip \labelsep {\bfseries
#2.}]}{\end{trivlist}}
\newenvironment{lemma}[2][Lemma]{\begin{trivlist}
\item[\hskip \labelsep {\bfseries #1}\hskip \labelsep {\bfseries #2.}]}{\end{trivlist}}
\newenvironment{sol}
    {\emph{Proof.}
    }
    {
    \qed
    }
\specialcomment{com}{ \color{blue} \textbf{Comment:} }{\color{black}} %for instructor comments while grading
\NewEnviron{probscore}{\marginpar{ \color{blue} \tiny Problem Score: \BODY \color{black} }}
%%%%%%%%%%%%%%%%%%%%%%%%%%%%%%%%%%%%%%%%%%%%%%%%%%%%%%%%%%%%%%%%%%%%%%%%%%%%%%%%%

\newcommand\restr[2]{{% we make the whole thing an ordinary symbol
  \left.\kern-\nulldelimiterspace % automatically resize the bar with \right
  #1 % the function
  \vphantom{\big|} % pretend it's a little taller at normal size
  \right|_{#2} % this is the delimiter
  }}





%%%%%%%%%%%%%%%%%%%%%%%%%%%%%%%%%%%%%%%%%%%%%
%Fill in the appropriate information below
\lhead{Trey Manuszak}  %replace with your name
\rhead{MAT 473: Intermediate Real Analysis II \\ Homework 12: 45, 46, 47, 48} %replace XYZ with the homework course number, semester (e.g. ``Spring 2019"), and assignment number.
%%%%%%%%%%%%%%%%%%%%%%%%%%%%%%%%%%%%%%%%%%%%%

\usepackage{blindtext}
\title{MAT 473: Intermediate Real Analysis II}
\date{April 24, 2020}
\author{Trey Manuszak\\ Arizona State University}


%%%%%%%%%%%%%%%%%%%%%%%%%%%%%%%%%%%%%%
%Do not alter this block.
\begin{document}
%%%%%%%%%%%%%%%%%%%%%%%%%%%%%%%%%%%%%%


\maketitle
\newpage


%Copy the following block of text for each problem in the assignment.
\begin{problem}{45}
  Recall that a function $f : \mathbb{R} \to \overline{\mathbb{R}}$ is \textit{measurable} (or \textit{Lebesgue measurable} if for every Borel set $E$ in $\overline{\mathbb{R}}$, we have that $f^{-1}(E)$ is a (Lebesgue) measurable set (in $\mathbb{R}$).) We say that $f$ is \textit{Borel measurable} if for every Borel set $E \subseteq \overline{\mathbb{R}}$, $f^{-1}(E)$ is a Borel set.

  Let $f : \mathbb{R} \to \mathbb{R}$ and $g : \mathbb{R} \to \overline{\mathbb{R}}$. Prove the following. 
  \begin{itemize}
    \item[(a)] If $f$ and $g$ are both Borel measurable, then $g \circ f$ is Borel measurable.
    
    \begin{sol}
      Suppose $f : \mathbb{R} \to \mathbb{R}$ and $g : \mathbb{R} \to \overline{\mathbb{R}}$ are both Borel measurable. Note, that for every $E \subseteq \overline{\mathbb{R}}$ Borel, 
      $$(g \circ f)^{-1}(E) = f^{-1}(g^{-1}(E)).$$
      Since $E$ is borel and $g$ is Borel measurable, then $g^{-1}(E)$ is Borel. Since $f$ is Borel measurable and $g^{-1}(E)$ is Borel, then $f^{-1}(g^{-1}(E))$ is Borel, which implies $(g \circ f)^{-1}(E)$ is Borel. Therefore, $g \circ f$ is Borel measurable.
    \end{sol}
    
    \item[(b)] If $f$ is measurable and $g$ is Borel measurable, then $g \circ f$ is measurable.
    
    \begin{sol}
      Suppose $f : \mathbb{R} \to \mathbb{R}$ is measurable and $g : \mathbb{R} \to \overline{\mathbb{R}}$ is Borel measurable. Similarly, for every $E \subseteq \overline{\mathbb{R}}$ Borel, 
      $$(g \circ f)^{-1}(E) = f^{-1}(g^{-1}(E)).$$
      Since $g$ is Borel measurable and $E$ is Borel, then $g^{-1}(E)$ is Borel. Since $f$ is measurable and $g^{-1}(E)$ is Borel, then $f^{-1}(g^{-1}(E))$ is measurable, which implies $(g \circ f)^{-1}(E)$ is measurable. Therefore, $g \circ f$ is measurable.
    \end{sol}
  \end{itemize}
  (It is a fact that there exists examples of measurable functions $f$ and $g$ such that $g \circ f$ is not measurable.)
\end{problem}

\begin{problem}{46}
  Let $f$ be a nonnegative simple function. Define a function $\mu : \mathcal{L} \to [0,\infty]$ by $\mu(E) = \int(f \cdot \chi_E)$. Prove that $\mu$ is \textit{countably additive}: if $E_1, E_2, \dots$ are pairwise disjoint measurable sets, then $\mu(\cup_{i = 1}^{\infty}E_i) = \sum_{i = 1}^{\infty}\mu(E_i)$.
\end{problem}

\begin{sol}
  For each $N > 0$, $\cup_{i = 1}^{N}E_i \subset \cup_{i = 1}^{\infty}E_i$, which implies $\mu \left( \bigcup_{j=1}^{N} E_i \right) \leq \mu \left( \bigcup_{i = 1}^{\infty} E_i \right)$. So, since each $E_i$ is disjoint and by finite subadditivity, we have $\sum_{i = 1}^{N} \mu(E_i) \leq \mu \left( \bigcup_{i = 1}^{\infty} E_i \right)$. Then, since $N$ does not determine the inequality, we have $\lim_{N \to \infty} \sum_{i = 1}^{N} \mu(E_i) = \sum_{i = 1}^{\infty} \mu(E_i) \leq \mu \left( \bigcup_{i = 1}^{\infty} E_i \right)$. Since countable subadditivity give the other direction of the inequality, we must have $\sum_{i = 1}^{\infty} \mu(E_i) = \mu \left( \bigcup_{i = 1}^{\infty} E_i \right)$, which implies $\mu$ is countably additive.
\end{sol}

\begin{problem}{47}
  Let $f$ be a nonnegative simple function. Prove that the following conditions are equivalent:
  \begin{itemize}
    \item[(a)] $\int f = 0$
    \item[(b)] $f = 0$ a.e.
    \item[(c)] Let $f = \sum_{i = 1}^{n}a_i\chi_{A_i}$ be any representation of $f$ with $a_i \geq 0$ for all $i$. For each $i$, if $a_i > 0$, then $m(A_i) = 0$.
  \end{itemize}

  \begin{sol}
    (a)$\Longrightarrow$(b): Suppose for some $X \in \text{dom}(f)$, $\int_X f = 0$. Let $\{x \in X : f(x) > 0\} = \cup_{n \in \mathbb{N}}\{x \in X : f(x) > \frac{1}{n}\}$. Then, 
    \begin{align*}
      m(\{x \in X : f(x) > \frac{1}{n}\}) &= \int_{\{x \in X : f(x) > \frac{1}{n}\}}1 \\
      &= n \int_{\{x \in X : f(x) > \frac{1}{n}\}} \frac{1}{n} \\
      &\leq n \int_{\{x \in X : f(x) > \frac{1}{n}\}} f \\
      &\leq n \int_{X} f \\
      &=0.
    \end{align*}
    Therefore, $m(\{x \in X : f(x) > \frac{1}{n}\}) = 0$, which implies $f = 0$ a.e.

    \vspace{1em}

    (b)$\Longrightarrow$(a): Let $A = \{x : f(x) = 0\}$ and $m(A^c) = 0$. Then, for $X \in \text{dom}(f)$, 
    \begin{align*}
      \int_X f &= \int_X f \cdot (\chi_A + \chi_{A^{c}}) \\
      &= \int_X f \cdot \chi_A + \int_X f \cdot \chi_{A^{c}} \tag*{(Since $A \cap A^c = \emptyset$)} \\
      &= \int_A f + \int_{A^c} f \\
      &= 0.
    \end{align*}

    \vspace{1em}

    (a)$\Longrightarrow$(c): Suppose that $\int_A f = 0$ where $A$ is a collection of disjoint sets. Thus, since $\int_E f = \int a_1\chi_{A_1} + \dots = 0$, then each term must be zero. That means that if $a_i > 0$ for some $i$, then $\chi_{A_i} = 0$, which implies $m(A_i) = 0$. 

    \vspace{1em}

    (c)$\Longrightarrow$(a): Suppose $f = \sum_{i = 1}^{n}a_i\chi_{A_i}$ be any representation of $f$ with $a_i \geq 0$ for all $i$. For each $i$, if $a_i > 0$, then $m(A_i) = 0$. Then, each term $a_i\chi_{A_i} = 0$ in the expansion of $f$, which implies $\int f = 0$.
  \end{sol}
\end{problem}

\begin{problem}{48}
  For $f,g : \mathbb{R} \to \mathbb{R}$ the \textit{join} of $f$ and $g$ is the function $f \vee g : \mathbb{R} \to \mathbb{R}$ defined by $$(f \vee g)(x) = \max\{f(x),g(x)\}$$ (i.e. the pointwise maximum of the two functions). The \textit{meet} is defined by $$(f \wedge g)(x) = \min\{f(x),g(x)\}.$$ The \textit{positive and negative parts} of $f$ are defined by $$f_+ = f \vee 0, \hspace{1em} f_- = -(f \wedge 0).$$ Prove the following.
  \begin{itemize}
    \item[(i)] If $f$ and $g$ are measurable then so are $f \vee g$ and $f \wedge g$.
    
    \begin{sol}
      Note, $\{x : (f \vee g)(x) > c\} = \{x : f(x) > c\} \cup \{x : g(x) > c\}$ and $\{x : (f \wedge g)(x) > c\} = \{x : f(x) > c\} \cap \{x : g(x) > c\}$. So, since the join and meet are a countable collection of union and intersected measurable sets, then the join and meet must also be measurable functions.
    \end{sol}

    \item[(ii)] $f_+ \geq 0,$ $f_- \geq 0$, and $f_+f_- = 0$.
    
    \begin{sol} There are the three following cases,
      $$f > 0 \longrightarrow f_+ > 0 \text{ and } f_- = 0 \Longrightarrow f_+f_- = 0,$$
      $$f < 0 \Longrightarrow f_+ = 0 \text{ and } f_- > 0 \Longrightarrow f_+f_- = 0,$$
      $$f = 0 \Longrightarrow f_+ = 0 \text{ and } f_- = 0 \Longrightarrow f_+f_- = 0.$$
      Therefore, in all cases, $f_+ \geq 0,$ $f_- \geq 0$, and $f_+f_- = 0$.
    \end{sol}

    \item[(iii)] $f = f_+-f_-$ and $|f| = f_++f_-$.
    
    \begin{sol}
      Focusing on the first part, if $f > 0$, then $f_+ - f_- = f - 0 = f$. If $f < 0$, then $f_+ - f_- = 0 - (-f) = f$. If $f = 0$, then $f_+ - f_- = 0 - 0 = 0 = f$. Therefore, $f = f_+ - f_-$. 

      Now, on the second part, if $f > 0$, then $f_+ + f_- = f + 0 = f = |f|$. If $f < 0$, then $f_+ + f_- = 0 - f = |f|$. If $f = 0$, then $f_+ + f_- = 0 + 0 = 0$. Therefore, $|f| = f_+ + f_-$.
    \end{sol}
    
    \item[(iv)] If $g,h \geq 0$ and $f = g - h$, then $g \geq f_+$ and $h \geq f_-$. Also, $g = f_+$ if and only if $h = f_-$, and this happens if and only if $gh = 0$.  
    
    \begin{sol}
      From (iii), we know $f = f_+ - f_-$. Suppose $f = g - h$ and $g,h \geq 0$. So, $g - h$ = $f_+ - f_-$. There are three cases. 
      
      \hspace{1em} One, if $f > 0$, then either $g > h$, which implies the difference $g - h$ and $f_+ - f_-$ must be the same, which implies $g > f_+$ and $h > f_-$ and $gh > 0$, or the differences are the same, which implies $g = f_+$ and $h = f_-$ and $gh = 0$. 
      
      \hspace{1em} Two, if $f < 0$ imples either $g < h$, which implies $g > f_+$ and $h > f_-$ to make up for the difference, which implies $gh > 0$, or, as in the last case, the differences are the same, which implies $g = f_+$ and $h = f_-$ and $gh = 0$. 

      \hspace{1em} Lastly, if $f = 0$, then $g = h = 0$ and $f_+ = f_- = 0$, which implies $g = f_+$ and $h = f_-$, which implies $gh = 0$.

      \hspace{1em} Therefore, we have if $g,h \geq 0$ and $f = g - h$, then $g \geq f_+$ and $h \geq f_-$. Also, $g = f_+$ if and only if $h = f_-$, and this happens if and only if $gh = 0$. 
    \end{sol}
  \end{itemize}
\end{problem}
%%%%%%%%%%%%%%%%%%%%%%%%%%%%%%%%%%%%%%%%
%Do not alter anything below this line.
\end{document}